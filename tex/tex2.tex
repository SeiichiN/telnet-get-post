\documentclass[uplatex, dvipdfmx]{jsarticle}

\include{begin}

\section{GETリクエスト(クエリ文字列)}

\subsection{英字の場合}

GETリクエストでクエリ文字列を送ることができる。

\fbox{ telnet 192.168.31.27 8080 } としたあとの画面で
以下を入力する。

\begin{lstlisting}
GET /get_receive.php?name=Taro HTTP/1.1
Host: 192.168.31.27
\end{lstlisting}

これは、サーバーのドキュメントルートにある get\_receive.php
をダウンロード要求している。そして、そのファイルに対して
\fbox{ name=Taro }という文字列を渡している。

サーバー側では、まず get\_receive.php が PHPプログラムとして
動作し、nameという名前で ``Taro'' という文字列を受け取る。
その後、PHPはHTML文字列を生成し、それをクライアントに返す。

以下のようなレスポンスとなる。

\begin{lstlisting}
HTTP/1.1 200 OK
Date: Sat, 10 Sep 2022 04:25:43 GMT
Server: Apache/2.4.54 (Debian)
X-Powered-By: PHP/7.4.30
Vary: Accept-Encoding
Content-Length: 354
Content-Type: text/html; charset=UTF-8

<!doctype html>
<html lang="ja">
  <head>
    <meta charset="utf-8"/>
    <meta name="viewport" content="width=device-width,initial-scale=1"/>
    <title>get_receive</title>
  </head>
  <body>
    <h1>get receive</h1>
    <p>name is Taro</p>
    <script>
    'use strict';

    </script>
  </body>
</html>
\end{lstlisting}

クライアントによって送られた ``name=Taro'' は、
\verb!<p>name is Taro</p>! というHTML文字列となって埋め込まれている。

\subsection{日本語文字列を送る場合}

GETリクエストにおいて、日本語文字列をクエリ文字列として送る場合は、
``URLエンコード''しなくてはならない。

たとえば、以下のサイトで URLエンコード できる。

\href{https://tech-unlimited.com/urlencode.html}{URLエンコード・デコード}

``桃太郎'' をURLエンコードすると、``\verb!%E6%A1%83%E5%A4%AA%E9%83%8E!''
となる。これを以下のように GETリクエストする。

\begin{lstlisting}[numbers=none]
GET /get_receive.php?name=%E6%A1%83%E5%A4%AA%E9%83%8E HTTP/1.1
Host: 192.168.31.27
 
\end{lstlisting}

以下のようなレスポンスが返ってくる。

\begin{lstlisting}
HTTP/1.1 200 OK
Date: Sun, 14 Aug 2022 14:02:37 GMT
Server: Apache/2.4.54 (Debian)
X-Powered-By: PHP/7.4.30
Vary: Accept-Encoding
Content-Length: 359
Content-Type: text/html; charset=UTF-8

<!doctype html>
<html lang="ja">
  <head>
    <meta charset="utf-8"/>
    <meta name="viewport" content="width=device-width,initial-scale=1"/>
    <title>get_receive</title>
  </head>
  <body>
    <h1>get receive</h1>
    <p>name is 桃太郎taro</p>
    <script>
    'use strict';

    </script>
  </body>
</html> 
\end{lstlisting}
\include{end}


実際は、ブラウザにて、日本語文字列をユーザーがいちいち
URLエンコードしなくてもいいように、ブラウザがURLエンコードして
サーバーにGETリクエストしてくれている。



%% 修正時刻: Sat 2022/09/10 12:12:430
