\documentclass[uplatex, dvipdfmx]{jsarticle}

\include{begin}

\section{POSTリクエスト}

\subsection{英字の場合}

POSTリクエストの場合、送信すべき文字列はボディ部の中に書く。
そして、何バイトなのかをヘッダ部に書く。
以下のようになる。

\begin{lstlisting}[numbers=none]
POST /post_receive.php HTTP/1.1
Host: 192.168.31.27
Content-Length:13
Content-Type: application/x-www-form-urlencoded

name=momotaro
\end{lstlisting}

今度は、ドキュメントルートの ``post\_receive.php''を指定している。
 ``name=momotaro''という文字列を送信している。13バイトである。

それと、フォームで送信するので、Content-Type: に
\verb!application/x-www-form-urlencoded! という記述が必要になる。

このようにフォームでは、ボディ部に送信文字列を記述しているのである。

これに対するレスポンスは以下である。

\begin{lstlisting}
HTTP/1.1 200 OK
Date: Sat, 10 Sep 2022 07:21:36 GMT
Server: Apache/2.4.54 (Debian)
X-Powered-By: PHP/7.4.30
Vary: Accept-Encoding
Content-Length: 360
Content-Type: text/html; charset=UTF-8

<!doctype html>
<html lang="ja">
  <head>
    <meta charset="utf-8"/>
    <meta name="viewport" content="width=device-width,initial-scale=1"/>
    <title>post_receive</title>
  </head>
  <body>
    <h1>post receive</h1>
    <p>name is momotaro</p>
    <script>
    'use strict';

    </script>
  </body>
</html>
\end{lstlisting}


\subsection{日本語文字列の場合}

日本語文字列の場合は、やはり URLエンコードをする。
以下は、桃太郎をURLエンコードして送信している。

\begin{lstlisting}
POST /post_receive.php HTTP/1.1
Host: 192.168.31.27
Content-Length:32
Content-Type: application/x-www-form-urlencoded

name=%E6%A1%83%E5%A4%AA%E9%83%8E
\end{lstlisting}

以下がそのレスポンスである。y

\begin{lstlisting}
HTTP/1.1 200 OK
Date: Sat, 10 Sep 2022 07:27:23 GMT
Server: Apache/2.4.54 (Debian)
X-Powered-By: PHP/7.4.30
Vary: Accept-Encoding
Content-Length: 361
Content-Type: text/html; charset=UTF-8

<!doctype html>
<html lang="ja">
  <head>
    <meta charset="utf-8"/>
    <meta name="viewport" content="width=device-width,initial-scale=1"/>
    <title>post_receive</title>
  </head>
  <body>
    <h1>post receive</h1>
    <p>name is 桃太郎</p>
    <script>
    'use strict';

    </script>
  </body>
</html>
\end{lstlisting}



\include{end}

%% 修正時刻: Sat 2022/09/10 12:12:430
